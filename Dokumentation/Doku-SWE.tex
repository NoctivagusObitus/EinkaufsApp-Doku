\documentclass[12pt,a4paper]{article}
\usepackage[utf8]{inputenc}
\usepackage{amsmath}
\usepackage{amsfonts}
\usepackage{amssymb}
\usepackage [ngerman] {babel}
\begin{document}
\title{Software Engineering Projekt}
\author{Gruppe Einkaufsapp}
\date {18.Oktober 2015}
\maketitle
\newpage
\tableofcontents
\newpage
\section*{Projektdokumentation}
\subsection*{Gruppenmitglieder}
\subsubsection*{Projektleiter}
Markus Hube
\subsubsection*{Entwicklung}
Sebastian Kiepsch
\newline
Michael Hein
\newline
Eric Sorgalla
\newline
Daniel Sawadenko 
\newline
Viktor Fuchs
\newline
Florian Schmitt 
\subsubsection*{Design}
Florian Graupeter
\newline
Moritz Karsten
\newline
Moritz Schaub
\newline
Jannis Grohs
\subsubsection*{Dokumentation}
Huong Dang
\newline
Thomas Elias
\newline
Annika Köstler
\newpage
% Hier beginnt nun die Einleitung.
\section*{Einleitung}
Diese Dokumentation soll einen näheren Einblick in den Umfang, 
den Nutzen, den Ablauf und das Ergebnis unseres Softwareprojekts 'Einkaufsapp' geben.  
\newline
Unsere entwickelte App dient dem Nutzer seine alltäglichen Einkaufserlebnisse, hinsichtlich seiner besuchten Läden und seiner gekauften Produkte zu dokumentieren und eine Übersicht über seine Finanzen zu erhalten.
Diese App soll gleichzeitig ein kleines Nachschlagewerk für die Nutzer darstellen, in der ein Überblick über Preis und Angebot bestimmter erstellt wird. 
Sie soll den täglichen Einkauf für alle Konsumenten erleichtern und es ermöglichen seine Ausgaben auf einen Blick zu sehen.
\newline
Diese Dokumentation dient als Nachschlagewerk für die Projektmitglieder und beschreibt alle während der Projektdurchführung aufgekommenen Ideen und Tasks sowie alle  erstellten Dokumente.
\newline
Zudem wurde eine Einteilung des Projektes in Definitionsphase, Planungsphase, Durchführungsphase und Abschlussphase als angemessen empfunden und in diesem Dokument angewandt.
Diese Dokumentation ist parallel zur Durchführungsphase entstanden.
\newline
\newpage
\section{Definitionsphase}
\subsection{Problembeschreibung}
Es gibt viele verschiedene Produkte, welche von unterschiedlichen Anbietern zu unterschiedlichen Preisen und Mengen angeboten werden. 
Da fällt es jedem Konsumenten schwer den Überblick zu behalten, welches Produkt bei welchen Anbieter am günstigsten ist und wie viel man in der Woche, im Monat oder im Jahr für ein bestimmtes Produkt bei einem bestimmten Anbieter ausgibt. Diese App soll dabei helfen den finanziellen Überblick zu behalten und eine Hilfe für alle Konsumenten sein, dsich öfter fragen, wo ihre Lieblingsprodukte am günstigsten angeboten werden und wie oft sie diese Produkte im Monat kaufen.
Zudem bietet die App z. B. WG-Mitgliedern die Möglichkeit die Ausgaben pro Person zu tracken was die manuelle Kalkulation am Ende eines Monats erspart. 

\subsection{Ursachenanalyse}
\subsection{Entscheidung zur Projektdurchführung}
\subsection{Projektziele}
\subsection{Projektorganisation}
\subsection{Kick-Off-Meeting}
\newpage
\section{Planungsphase}
% Hier soll alles zur Planung stehen.
\subsection{Aktivitätsliste}
\subsubsection{Designer}
\subsubsection{Entwickler}
\subsection{Qualitätssicherung}
\subsubsection{Logo- und Designentwicklung}
\subsubsection{Websites}


\newpage
\section{Durchführungsphase}
%Hier soll die genaue Durchführung beschrieben werden.
\subsection{Designer}
\subsection{Entwickler}
\subsection{Organisationstools}
Wir haben GitHub als allgemeine Ablage genutzt und jedes Teammitglied hatte darauf Zugriff.


\newpage
\section{Abschlussphase}
%Hier wird das Ergebnis beschrieben.
\subsection{Resumee}
\subsection{Fazit}
\newpage
\section*{Quellen}
\newpage
\section*{Anhang}
\newpage
\section*{Glossar}

\end{document}