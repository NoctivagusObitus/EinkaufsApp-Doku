\documentclass[12pt,a4paper]{article}
\usepackage[utf8]{inputenc}
\usepackage{amsmath}
\usepackage{amsfonts}
\usepackage{amssymb}
\usepackage [ngerman] {babel}
\usepackage{hyperref}
\usepackage{array}
\begin{document}
\title{Software Engineering Projekt}
\author{Gruppe Einkaufsapp}
\date {18.Oktober 2015}
\maketitle
\newpage
\tableofcontents
\newpage
\newpage
\section*{Abkürzungsverzeichnis}
\newpage
\section*{Tabellenverzeichnis}
\newpage
\section*{Projektdokumentation}
\subsection*{Gruppenmitglieder}
\subsubsection*{Projektleiter}
Markus Hube
\subsubsection*{Entwicklung}
Sebastian Kiepsch
\newline
Michael Hein
\newline
Eric Sorgalla
\newline
Viktor Fuchs
\newline
Florian Schmitt 
\subsubsection*{Design}
Florian Graupeter
\newline
Moritz Karsten
\newline
Moritz Schaub
\newline
Jannis Grohs
\newline
Daniel Sawadenko 
\subsubsection*{Dokumentation}
Huong Dang
\newline
Thomas Elias
\newline
Annika Köstler
\newpage

%Hier beginnt nun die Einleitung.

\section*{Einleitung}
\addcontentsline{toc}{section}{Einleitung}
Diese Dokumentation soll einen näheren Einblick in den Umfang, den Nutzen, den Ablauf und das Ergebnis unseres Softwareprojekts 'EinkaufsApp' geben.  
\newline
Die EinkaufsApp dient dem Nutzer seine alltäglichen Einkaufserlebnisse, hinsichtlich der besuchten Läden und gekauften Produkte zu dokumentieren und eine Übersicht über seine Finanzen zu erhalten.
Gleichzeitig soll sie als kleines Nachschlagewerk fungieren, welches Überblick über Preis und Angebot bestimmter Produkte bietet.
Der alltägliche Einkauf wird hinsichtlich des Monitoring der Finanzen und Produktauswahl aufgrund der Funktionalitäten der EinkaufsApp erleichtert.
\newline
Die Dokumentation umfasst die kompletten Phasen der Vorbetrachtung, Planung und Entwicklung der EinkaufsApp mit den jeweiligen Ideen, Tasks und angefertigten Dokumenten und fungiert als Leitfaden für alle Projektmitglieder durch das gesamte Projekt.
\newline
Zudem wurde eine Einteilung des Projektes in Definitionsphase, Planungsphase, Durchführungsphase und Abschlussphase als angemessen empfunden und in diesem Dokument angewandt.
Diese Dokumentation ist parallel zur Durchführungsphase entstanden.

\newpage
\section{Vorbetrachtung}
Die Vorbetrachtung beinhaltet alle vorbereitenden Aktivitäten, die vor der Entwicklung der Applikation getätigt wurden. Dazu gehören die konkrete Problembeschreibung, der darauffolgende Lösungsansatz und die Zielsetzungen für die Umsetzung der Entwicklung.


\subsection{Problembeschreibung}
Die steigende Vielfalt an Produkten und die Preisschwankungen der Anbieter führen den Konsumenten zu einer Unübersichtlichkeit über die Angebotsvielfalt und der damit verbundenen Ausgaben.
Wie in der Einleitung beschrieben, bietet die EinkaufsApp die Möglichkeit einen Überblick über getätigte Einkäufe zu schaffen,
um vor allem die finanziellen Ausgaben pro Woche, Monat oder Jahr zu tracken und Preise gleicher Produkte von unterschiedlichen Anbietern zu vergleichen.  
Diese App soll zudem noch dabei helfen den finanziellen Überblick zu behalten und eine Hilfe für alle Konsumenten sein, die sich öfter fragen, wo ihre Lieblingsprodukte am günstigsten angeboten werden und wie oft sie diese Produkte im Monat kaufen.
Zusätzlich gibt es eine Gruppenfunktion, die bestimmten angelegten Gruppen, z.B WG-Mitgliedern, die Möglichkeit bietet, die Ausgaben pro Person zu tracken, was die manuelle Kalkulation am Ende eines Monats erspart. 

\subsection{Entscheidung zur Projektdurchführung}
Die EinkaufsApp soll die EANs (European Article Number) der Produkte, die Konsumenten bei ihren Einkäufen in den Warenkorb legen und bezahlen, speichern.
Sie soll es zudem ermöglichen die Preise der Produkte und die damit verbundenen Kosten auf Gruppen oder einzelne Personen aufzuteilen und im Ergebnis eine finanzielle Auswertung aufzeigen.
Unser Ziel mit der Umsetzung des Projektes ist es, eine App zu entwickeln, die eine Lösung für das in der Problemstellung genannte Problem darstellt. 
Die Produktvielfalt der verschiedenen Produktanbieter wird vereinfacht dargestellt, der Konsument sieht auf einen Blick seine Ausgaben und Gruppenmitglieder müssen nicht noch manuell nach dem Einkauf jegliche Preise zusammenrechnen.

\newpage

\subsection{Funktionen}
Dieses Kapitel beinhaltet die geplanten Funktionen der Applikation. Die Unterteilung erfolgt in die Hauptteile Einkauf, Auswertung und Gruppenverwaltung. Es werden jeweils pro Kategorie die Hauptfunktionen, also die Funktionen die implementiert werden müssen, Zusatzfunktionen, also die Funktionen die nachdem die Hauptfunktionen umgesetzt wurden implementiert werden und den Ideen, die aber aus Kapazitätsgründen nicht umgesetzt werden, aber in Zukunft umgesetzt werden können.
\subsubsection{Einkauf}
\paragraph{Hauptfunktion}
\begin{itemize}
\item[-] Grundlegend muss die App die Funktion der Erstellung von Einkaufslisten haben. Hierzu gehören auch die Artikelaufnahme in diese Liste sowie die nachfolgende Bearbeitung dieser Liste bei Änderungsbedarf des Nutzers. Zum Schluss muss sie den kompletten Einkauf zusammenfassen. Der Einkauf wird dann abgeschlossen sobald der Nutzer dies auch bestätigt.
\end{itemize}
\paragraph{Zusatzfunktion}
\begin{itemize}
\item[-]Zusätzlich soll die App die Funktion des Preisvergleichs von Artikeln in unterschiedlichen Märkten besitzen.
\end {itemize}
\paragraph{Ideen Out-Of-Scope}
\begin{itemize}
\item[-]Es soll keine weitere Zusatzfunktion implementiert werden
\end{itemize}
\subsubsection{Markt}
\paragraph{Hauptfunktion}
\begin{itemize}
\item[-]Bevor der Einkaufsprozess gestartet wird soll die App die Funktion der Standortbestimmung haben. Sobald der Nutzer vor einem Markt steht ermittelt die App via GPS seinen Standort. Falls der Markt nicht gefunden wird, soll es die Möglichkeit der Hinzufügung eines neuen Marktes geben. Der Nutzer gibt dann hier die Daten des neuen Marktes an.
\end{itemize}
\paragraph{Zusatzfunktion}
\begin{itemize}
\item[-]Es werden keine weiteren Zusatzfunktionen implementiert.
\end{itemize}
\paragraph{Ideen Out-of-Scope}
\begin{itemize}
\item[-]Ein Markt kann über die App von einem Nutzer z. B. durch ein 5-Sterne Bwertungssystem bewertet werden
\end{itemize}

\subsubsection{Benutzerprofil}
\paragraph{Hauptfunktion}
\begin{itemize}
\item[-]Der Nutzer kann die App erst nutzen, wenn dieser ein Benutzerprofil erstellt. Das Profil besteht grundlegend aus Namen, E-Mailadresse und einem Passwort. Die von ihm getätigten Einkäufe sind dann eindeutig zuordbar.
\end{itemize}
\paragraph{Zusatzfunktion}
\begin{itemize}
\item[-]Es werden keine weiteren Zusatzfunktionen implementiert.
\end{itemize}
\paragraph{Ideen Out-of-Scope}
\begin{itemize}
\item[-]Der Nutzer kann sich auch mit seinem FB-Profil oder via Twitter anmelden.
\end{itemize}
\subsubsection{Gruppeneinkäufe}
\paragraph{Hauptfunktion}
\begin{itemize}
\item[-]Einkäufe können einzeln Gruppen und Personen zugeordnet werden.
\end{itemize}
\paragraph{Zusatzfunktion}
\begin{itemize}
\item[-]Es werden keine weiteren Zusatzfunktionen implementiert.
\end{itemize}
\paragraph{Ideen Out-of-Scope}
\begin{itemize}
\item[-]Es wurden keine weiteren noch nicht-umsetzbare Funktionen formuliert.
\end{itemize}
\subsubsection{Auswertung}
\paragraph{Hauptfunktion}
\begin{itemize}
\item[-]Der Nutzer kann vergangene Einkäufe auswerten lassen. 
\\Folgende Unterscheidungen werden gemacht:
\begin{itemize}
\item[a)]Kosten pro Zeitraum
\item [b)] Kaufhäufigkeit eines Artikels und die dazugehörigen Gesamtkosten
\item[c)] Kosten pro Artikelkategorie
\item[d)] monetäre Ausgaben je Käufergruppe
\end{itemize}
\end{itemize}

\newpage
\subsection{Projektorganisation}
Die Projektgruppe der EinkaufsApp teilte sich am 02. Oktober 2015 in die Untergruppen Dokumentation, Design und Entwicklung auf.
Der Projektleiter und in diesem Falle auch Projektmanager wurde ebenso an diesem Tag ernannt.
Als Projektmanager war er nun für die Team- und Projektorganisation zuständig, wozu das Einhalten der Projekt- und Meilensteinplanung  und das Erfüllen der Projektziele hoch priorisiert wurden.
Jegliche Unterhaltung basierte auf Mailverkehr oder fand durch Telefonkonferenzen statt. Jede Untergruppe musste sich selbst organisieren und wöchentlich ein Update dem Projektleiter zukommen lassen. Jeden Montag fanden Status-Telekonferenzen statt, wo sich alle Teammitglieder zusammen fanden und über den aktuellen Stand der Untergruppen informierten und über aufgekommene Probleme diskutierten. Die einzelnen Aktivitäten der Untergruppen werden in den Unterpunkten  1.5 - 1.7 noch genau erläutert.


\subsubsection{Kick-Off-Meeting}
Am 02. Oktober  2015 fand das erste Meeting mit der gesamten Projektgruppe statt. In diesem Kick-Off-Meeting traf man Absprachen über das weitere Vorgehen und die Projektumsetzung der Ideen und Ziele, welche im Kapitel 1.3 ausführlich beschrieben wurden. Es wurde über die weitere Kommunikationsform abgestimmt und ebenso fest gelegt, dass wöchentlich Telefonkonferenzen innerhalb der Untergruppen zum weiteren Vorgehen des Projektes statt finden werden, sodass die genannten Projektziele bis zum festgelegten Datum umgesetzt werden können. 
Die Untergruppen einigten sich außerdem auf Tools, die effizient und sinnvoll zur Umsetzung der anstehenden Aktivitäten und zum Einhalten der Projektziele verwendet wurden. Die einzelnen Tools der Untergruppen werden ausführlich im Abschnitt Organisationstools aufgezählt und definiert.
\newpage

\subsubsection{Soll-Analyse}
In dem hier angeführten Kapitel werden konkrete Ziele für das bevorstehende Projekt formuliert, die auf den oben aufgeführten Funktionen der Applikation basieren. (Tabelle der Grundfunktionen, Die Umsetzungsmöglichkeiten, Prioritäten)

\begin{tabular}{|l|c|r|}
\hline
 Funktion & nötige Umsetzungsaspekte & Priorität \\
\hline
Datenbank & Aufstellen einer MongoDB & Prio 1\\
\end{tabular}

\begin{itemize}
 \item[1.1)] 
  Die App soll in Echtzeit die monetären Ausgaben einer Person speichern, sowie ausgewertet wiedergeben.
 \item[1.2)] Hierfür soll es möglich sein:
 \begin{itemize}
 \item[a)] bei einem Einkauf Informationen über einen Artikel von einem Etikett via Barcodescanner einzulesen, beziehungsweise bei bestehender EAN Nummer aus einer Datenbank zu laden und aus diesen Argumenten einen Einkauf zu erstellen
  \item[b)] sonstige Kosten aufzunehmen, die nicht mit einem EAN Code in Verbindung gebracht werden können.
  \item[c)] von aktiven Einkäufen unabhängige, regelmäßige Kosten zu erfassen.
  \end{itemize}
 \item[1.3)] Es soll möglich sein für jemand anderen oder eine Gruppe (z.B. WG) einzukaufen.
 \item[1.4)] Die Daten werden zentral in einer, über das Internet erreichbare, Datenbank gespeichert.
 \item[1.5)] Die App soll primär ein einfaches Front End bereitstellen, um Informationen zu sammeln und zu verwalten
 \item[1.6)] Eine Web Site ist momentan out of scope, wäre aber eine sinnvolle Ergänzung für die Ausgabe von Statistiken und die Benutzerverwaltung.
 \item[1.7)] Die Möglichkeiten der Auswertung sind vielfältig und können in Listen oder Diagrammen dargestellt werden.
 \item[1.8)] Auswertungsbeispiele:
 \begin{itemize}
\item[a)]Ausgaben innerhalb eines bestimmten Zeitraumes (z.B. Woche oder Monat)
 \item[b)] Maximal oder Minimalpreis innerhalb eines Zeitraumes (z.B. Woche oder Monat) 
\item[c)] Eine Grafik, die den Ausgabenverlauf innerhalb eines Zeitraumes darstellt
\item[d)] Eine Extrapolation regelmäßig gekaufter Artikel (Ersatz des „Einkaufzettels“)
\item[e)]Das persönliche Tracking der allgemeinen Ausgaben
\end{itemize}
 \item[1.9)] Außerdem nicht personenbezogene Auswertungen:
 \begin{itemize}
\item[a)]über beliebteste Artikel
\item[b)] beliebteste Märkte
\item[c)] Durchschnittspreise eines Artikels
\end{itemize}
 \item[2.1)] Ein online verfügbarer Server, auf dem seinerseits ein Datenbank Server und ein Web Server läuft
 \begin{itemize}
\item[a)]Als Datenbank-Server wird MongoDB verwendet
\item[b)]Als Web Server wird Apache verwendet
\end{itemize}
\item[2.2)] Auf dem Web Server befindliche PHP-Skripte stellen die Verbindung zur Datenbank her.
\item[2.3)] Aus der Android-App heraus wird mittels HTTP-Post eine Anfrage an die PHP-Skripte geschickte und die Antwort im JSON-Format wieder an die App zurück geschickt.
\end{itemize}
\newpage

\subsubsection{Ist-Analyse}
Zu Beginn wurden die jeweiligen Kompetenzen der Projektmitarbeiter zum Standpunkt vor der Durchführung des Projektes niedergeschrieben. Davon leiteten sich die Zugehörigkeiten jeder einzelnen Person in die Gruppen ab.\\
\newline

\begin{tabular}{|m{5cm}|m{5cm}|m{5cm}|}
\hline
\textbf {Name} & \textbf {Skills} & \textbf {Teamzuordnung} \\
\hline
\centering Annika Köstler & \begin {itemize}
\item Zwei Jahre Arbeit im Controlling
\item Tools:LaTex
\item Grundkenntnisse VBA
\item Protokollierung von Meetings
\end{itemize}
& Annika Köstler wird aufgrund ihrer Kompetenzen im Bereich Protokollierung in der Gruppe Dokumentation arbeiten.
\\
\hline
\centering Eric Sorgalla & 
\begin {itemize}
\item Grundkenntnisse (Java, C/C++, Javascript, HTML/CSS, VBA, SQL)
\item 1 Jahr Projektleitung ISIPT (nur kaufmännische Verantwortung) 
\end {itemize}
& Eric Sorgalla wird aufgrund seiner Programmiererfahrung bei den Entwicklern arbeiten

\\
\hline
\centering Huong Dang & \begin {itemize}
\item Zwei Jahre Vertrieb
\item Tools LaTex
\item Grundlagen VBA 
\end {itemize}
& Aufgrund der regelmäßigen Quality Check Aufgaben im Betrieb arbeitet Huong Dang in der Gruppe Dokumentation.

\\
\hline
\centering Jannis Grohs & \begin {itemize}
\item Datenbanken (MYSQL, Apex)
\item Programmiererfahrung (VBA, JAVA, Apex)
\item Projektmanagement 
\item Design - und Marketingtechniken
\end {itemize}
& Jannis Grohs wird aufgrund seines Know-Hows für das kontinuierliche Quality Check der Entwicklung und Designs zuständig sein.

\\
\hline
\centering Markus Hube & \begin {itemize}
\item Zwei Jahre PMO der operational services
\item Zwei Jahre Programmiererfahrung (VBA)
\item Bereits Vorarbeit zum Thema EinkaufsApp geleistet 
\end {itemize}
& Da Markus Hube bereits zum Thema EinkaufsApp Vorarbeit geleistet und Erfahrung im Projektmanagement gesammelt hat, übernimmt er die Position des Projektleiters.

\\
\hline
\centering Michael Hein & \begin {itemize}
\item Zwei Jahre Applikations Administration
\item Java Erfahrung
\item VBA Erfahrung
\item Skript Programmierung
\end {itemize}
& Durch Michael Heins langjähriger Programmiererfahrung wird er bei den Entwicklern tätig sein.

\\
\hline
\centering Moritz Karsten & \begin {itemize}
\item  Zwei Jahre Projektansprechpartner Messe Berlin
\item  Application Management
\end {itemize}
& Durch seine Erfahrung im Bereich Prozessablauf und Konzeptentwicklung wird Moritz Karsten bei der Gruppe Design arbeiten.
\\
\hline
\centering Moritz Schaub & \begin {itemize}
\item  Zwei Jahre Co-Product Owner in iOS und Android Messaging Produkt in AGILER Entwicklung (internationales, crossfunktionales Team)
\item  Erstellung von komplexen Prototypen mit Proto.io auf Basis von HTML5
\item Durchführung von Design Thinking Workshops
\end {itemize}
& Da Moritz Schaub Vorkenntnisse im Bereich Prototyperstellung von Apps hat, wird er für das visuelle Design zuständig sein.

\\
\hline
\centering Thomas Elias & \begin {itemize}
\item gute Kenntnisse in Projektkoordination
\item Anforderungen und Arbeitspakete definieren
\item Erfahrungen Customizing von Dokumenten-Layouts
\item Kommunikation zwischen versch. Abteilungen zum Transparent-Machen der Informationen
\end {itemize}
& Aufgrund seiner guten Kenntnisse im Bereich Projektkoordination wird Thomas Elias in der Dokumentation arbeiten und für die Meilensteinplanung verantwortlich sein.

\\
\hline
\centering Victor Fuchs & \begin {itemize}
\item  gute Excelkenntnisse
\item  gute Kenntnisse im Rechnungswesen und Controlling
\end {itemize}
& Victor Fuchs wird in der Gruppe Entwicklung arbeiten und hierbei die benötigten Konzeptdiagramme erstellen.

\end{tabular}
\newline
\newline

Insgesamt gibt es demnach drei Designer, fünf Entwickler und drei Gruppenmitglieder, die für die Dokumentation verantwortlich sind. Zudem sind zwei Gruppenmitglieder für das stetige Quality Check bei den Entwicklern und Designern zuständig.

Der Projektleiter Markus Hube hat sich bevor das Projekt gestartet ist ein Grobkonzept (ER-Modell) erstellt um die groben Anforderungen der App einzugrenzen. Diese Übersicht dient als Grundlage für die Entwickler, die im Nachhinein noch ausgebaut wird.
\newpage

\subsubsection{Anforderungsanalyse}
Was benötigen wir? 
\newpage

\subsubsection{Arbeitsplanung}
\textbf{Designer:}
\textbf{Florian Graupeter:}
\begin{itemize}
\item[-]Konzeption Grundstruktur
\item[-]Wie ist der Ablauf bei der App Nutzung, welche Sonderfälle müssen an welchen Punkten beachtet werden, wie ist die generelle Struktur
\end{itemize}
\textbf{Moritz Karsten:}
\begin{itemize}
\item[-]Konzeption Funktionalitäten und Informationsfluss einzelner Ansichten
\item[-]welche Knöpfe soll es geben, welche Informationen werden angezeigt, welche Informationen werden zwischen zwei Ansichten ausgetauscht
\end{itemize}
\textbf{Moritz Schaub:}
\begin{itemize}
\item[-]Visuelles Design
\item[-]Farben, Formen und Anordnung, von Butten und Feldern 
\item[-]eventuelles Logodesign + neuer Name für App
\end{itemize}
\textbf{Jannis Grohs:}
\begin{itemize}
\item[-]Quality of Service (Konsistenz in allem beachten und Vorgaben einhalten)
\item[-]Zusammenfassung über Erwartungshorizont und Ausarbeitung zu Fragen
\item[-]Einfachheit und Intuition im Design beachten, zuverlässige Fehlerbehandlung beachten
\end{itemize}
ToDos für alle Designer:
\begin{itemize}
\item[-]Flussdiagramm erstellen und erweitern
\item[-]Erwartungshorizont der App definieren
\item[-]Anfangsstruktur für die Entwickler festlegen
\end{itemize}
\newpage

\textbf{Entwickler:}
Hier fehlen leider noch die einzelnen Aufgabenbereiche von Vielen!  Unbedingt nachtragen!
\newline
\textbf{Sebastian Kiepsch:}
\begin{itemize}
\item[-] Einführung in die Technik- Präsentation vorbereiten
\item[-]Modulplan erstellen
\item[-] bereits erstellten Code kommentieren
\end{itemize}
\textbf{Michael Hein:}
\begin{itemize}
\item[-]Erstellung einer Benutzerverwaltung mit User Interface
\item[-] bereits erstellten Code kommentieren
\end{itemize}
\textbf{Eric Sorgalla:}
\begin{itemize}
\item[-] Einführung in die Technik- Präsentation vorbereiten
\item[-] Modulplan erstellen
\item[-] Kanban-Board erstellen
\end{itemize}
\textbf{Viktor Fuchs:}
\begin{itemize}
\item[-] Namenskonventionen ausarbeiten - Variablen und Funktionen
\end{itemize}
\textbf{Florian Schmitt:}
\begin{itemize}
\item[-] HTML-Code der Proto.io- App exportieren und durchgehen
\end{itemize}
\textbf{offene Aufgaben:}
\begin{itemize}
\item[-] aktualisiertes DB Modell erstellen
\item[-] EAN-Kategorien recherchieren und Tabelle füllen
\item[-] neues DB Modell implementieren
\item[-] UML Diagramm erstellen
\end{itemize}

\newpage
\textbf {Dokumentation:}
\newline
\textbf{Thomas Elias:}
\begin{itemize}
\item[-] Projektplan erstellen 
\item[-] Skillliste der einzelnen Projektmitglieder erstellen
\item[-] Aktivitätsliste aller Projektmitglieder definieren
\item[-] Ausarbeiten allgemeiner Informationen für Handbuch
\end{itemize}

\textbf{Huong Dang:}
\begin{itemize}
\item[-] Projektseitige Dokumentation
\item[-] Dokumentation der Emails, Meetings und Telefonkonferenzen
\item[-] Ausarbeiten der Aktivitäten des Einkaufsprozesses für Handbuch 
\end{itemize}

\textbf{Annika Köstler:}
\begin{itemize}
\item[-] Schreiben der endgültigen Dokumentation
\item[-] Ausarbeiten der Installationsanleitung 
\end{itemize}
\newpage

\subsubsection{Scrum}
Rahmenbedingungen für Profil und Einkaufslisten
\newpage
\subsection{Sicherheit}
\newpage


\section{Systemarchitektur/-landschaft}
\newpage
\subsection{Modell, View, Controller}
\newpage
\subsection{MongoDB}
\newpage
\subsubsection{OpenShift Server}
\newpage
\subsubsection{Basis Tools}
\newpage

\section{Durchführungsphase}
In diesem Abschnitt wird erklärt, wie die EinkaufsApp nach und nach aufgebaut wurde und insbesondere werden die einzelnen Punkte wie Registrierung, Login, Einkauf, Nutzerverwaltung und die Auswertung beschrieben.
(Idee: Roadmap, s. Flussdiagramm) (Was haben die Designer für die App geplant und was können die Entwickler davon umsetzen) -> nachfolgende Struktur erklären
\subsection*{Einleitung}
\newpage
\subsection{Systemarchitektur}
%Architekturbild
\subsection{Modell-View-Controller}
%Wie ist es aufgebaut?
%Zeichnung dafür hat Erik 
%Programmiersprache
\subsubsection{MongoDB}
%Development --> v3 dia als aktuelles Datenmodell
%Basti erstellt Dokument
\subsubsection{OpenShift Server}
%bastelt Basti und schickt es uns
\subsubsection{Tools}
\newpage
\subsection{Registrierung}
Bei der EinkaufsApp muss sich jeder Nutzer mittels einer Email-Adresse und einem Kennwort bei der App registrieren.
Eine Registrierung ist bei dieser App unentbehrlich, da für jeden Nutzer ein Profil angelegt wird und er mit diesem Profil seine Produkte einscannen kann und seine Finanzen im Blick behalten kann.
In den drei folgenden Unterpunkten wird der Prozess der Registrierung aus Seiten der Designer, Entwickler und der Dokumentation beschrieben.
\subsubsection*{Design}
% Flussdiagramm Login 
\subsubsection*{Entwicklung}
Die Entwickler befassen sich mit den Funktionen der Applikation  und sorgen bei der Registrierung dafür, dass alle Daten ordentlich geprüft und in die Datenbank eingepflegt werden.
Als Datenbank wird die MongoDB genutzt und via RoboMongo gemanagt. Als Programmiersprache wird in allen, auch den folgenden Unterpunkten JavaScript verwendet.
Bei der Registrierung werden die einzelnen Benutzereingaben durch bestimmte Regeln auf z.B. Länge, Email-Format, Eindeutigkeit, sowie Sicherheitskriterien der Passwortsvergabe in der Applikation geprüft.
% Konkrete Regeln siehe Handbuch
Wenn alle Prüfungen erfolgreich waren, wird der Benutzer angelegt und das Passwort verschlüsselt in Form eines Hashs in der Datenbank gespeichert.
\subsubsection*{Problembeschreibung} --> Probleme beschreiben
Eine ganz wichtige und entscheidende Rolle spielt hierbei der Datenschutz und die Datensicherheit. Alle eingetragenen Daten des Nutzers müssen abgesichert sein, sodass kein Dritter sich an diesen Daten vergreifen kann.
\newpage
\subsubsection{Login}
%FLussdiagramm-Ausschnitt
\subsubsection*{Entwicklung}
Die Entwicklung beschäftigt sich mit der Funktionsweise des Logins. und prüft hierbei, ob der Benutzer in der Datenbank existiert und falls dies der Fall ist, ob das dazugehörige Passwort korrekt ist. Stimmen dann Benutzername und Passwort überein, wird der Benutzer eingeloggt. 
Wenn der Benutzer sein Passwort vergessen hat, kann er dieses zurücksetzen lassen. Hierbei bekommt er eine E-Mail an die im Userprofil hinterlegte E-Mail Adresse. Diese enthält ein Token womit es einen Benutzer ermöglicht wird sein Passwort zu ändern. Dieses Token ist genau eine Stunde gültig, danach verfällt es. 
Nachdem der Nutzer sein Passwort zweimal eintragen musste, ändert die Datenbank das Kennwort des Nutzer und speichert dieses.
\subsubsection*{Problembeschreibung} --> Probleme beschreiben
Fehler bei Passwortrücksetzung weil Email nicht verschickt wird.
\newpage


\subsection{Einkauf}
Gruppeneinkauf vereinfacht, Einkauf fortsetzen eingebaut und Passwort rücksetzen
Gruppenverwaltung wurde zu Verwaltung allgemein:  Verwaltung von Gruppen UND KONTAKTE 
Kontaktverwaltung wird noch erstellt"
\subsubsection*{Design}
\subsubsection*{Entwicklung}
\subsubsection*{Dokumentation} --> Probleme beschreiben
\newpage

\subsection{Nutzerverwaltung}
\subsubsection*{Design}
\subsubsection*{Entwicklung}
\subsubsection*{Dokumentation} --> Probleme beschreiben
\newpage

\subsection{Auswertung}
\subsubsection*{Design}
\subsubsection*{Entwicklung}
\subsubsection*{Dokumentation} --> Probleme beschreiben
\newpage

\section{Problemzusammenfassung}
%Hier wird das Ergebnis beschrieben.
\subsection{Usability der App}
\newpage
\subsection{Organisation Projektmanagement}
\newpage
\section{Projektabschluss}
\subsection{Fertiges Produkt}
\subsection{Aussichten}
nicht umgesetzte Ideen --> siehe Excelliste
\newpage
\subsection{Zusammenfassung}
\newpage
\section{Lesson learned}
\newpage
\section*{Quellen}
\addcontentsline{toc}{section}{Quellen}
\subsection*{Internetquellen}
\begin{itemize}
\item[1.]Ionic Framework: \url{http://ionicframework.com/}
\item[2.]Ionic Guide: \url{http://ionicframework.com/docs/guide/}
\item[3.]Ionic Getting Started: \url{http://ionicframework.com/getting-started/}
\item[4.]ngCordova - Plugin Seite \url{http://ngcordova.com/}
\item[5.]BarCode Scanner : Plugin \url{hhttp://ngcordova.com/docs/plugins/barcodeScanner/}
\item[6.]Beispiel Projekt: \url{https://github.com/bastisk/suedm}
\item[7.]Editor: \url{http://brackets.io/}
\item[8.]Angular JS-Kurs: \url{https://www.codeschool.com/courses/shaping-up-with-angular-js/}
\item[9.]Tutorial zum Routing: \url{https://scotch.io/tutorials/angular-routing-using-ui-router}
\item[10.]App-Projekt: \url{http://www.mobile2b.de/ablauf-app-projekt/}
\item[11.] Dokumentationshilfe: \url{http://www.tellsbells.de/dokuwebsite/tbdokumentation.pdf}
\item[12.] Dokumentationshilfe: \url{https://www.lecturio.de/magazin/projekte-dokumentieren/}
\item[13.] Open Source mit API über eine einfachen HTTP-GET-Reguest: \url{http://www.opengtindb.org/api.php}
\item[14.] Suchmaschine der Firma die GTIN-Nummern verwaltet: \url{http://www.gepir.de/v31/V31_client/gtin.aspx}
\end{itemize}

\newpage
\section*{Organisationstool- Übersicht}
\begin{itemize}
\item[-]Allgemeine Ablage: GitHub
\item[-]Diskussionsrunden: Slack
\item[-]Informationsaustausch: via Email
\item[-]Diagramme zeichen: via Dia 
\item[-]Kreieren von Web-Prototypen: proto.io
\item[-]Datenbanken und Datenbankenverwaltung: MongoDB, RoboMongo
\end{itemize}
\newpage
\section*{Anhang}
\addcontentsline{toc}{section}{Anhang}

\end{document}

    Status API Training Shop Blog About Pricing 

    © 2015 GitHub, Inc. Terms Privacy Security Contact Help 
