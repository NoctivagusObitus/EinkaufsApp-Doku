\documentclass[12pt,a4paper]{article}
\usepackage[utf8]{inputenc}
\usepackage{amsmath}
\usepackage{amsfonts}
\usepackage{amssymb}
\usepackage [ngerman] {babel}
\usepackage{hyperref}
\begin{document}
\title{Software Engineering Projekt}
\author{Gruppe Einkaufsapp}
\date {18.Oktober 2015}
\maketitle
\newpage
\tableofcontents
\newpage
\section*{Projektdokumentation}
\addcontentsline{toc}{section}{Projektdokumentation}
\subsection*{Gruppenmitglieder}
\addcontentsline{toc}{subsection}{Gruppenmitglieder}
\subsubsection*{Projektleiter}
Markus Hube
\subsubsection*{Entwicklung}
Sebastian Kiepsch
\newline
Michael Hein
\newline
Eric Sorgalla
\newline
Viktor Fuchs
\newline
Florian Schmitt 
\subsubsection*{Design}
Florian Graupeter
\newline
Moritz Karsten
\newline
Moritz Schaub
\newline
Jannis Grohs
\newline
Daniel Sawadenko 
\subsubsection*{Dokumentation}
Huong Dang
\newline
Thomas Elias
\newline
Annika Köstler
\newpage

% Hier beginnt nun die Einleitung.

\section*{Einleitung}
\addcontentsline{toc}{section}{Einleitung}
Diese Dokumentation soll einen näheren Einblick in den Umfang, 
den Nutzen, den Ablauf und das Ergebnis unseres Softwareprojekts 'Einkaufsapp' geben.  
\newline
Unsere entwickelte App dient dem Nutzer seine alltäglichen Einkaufserlebnisse, hinsichtlich seiner besuchten Läden und seiner gekauften Produkte zu dokumentieren und eine Übersicht über seine Finanzen zu erhalten.
Gleichzeitig soll sie ein kleines Nachschlagewerk für Nutzer darstellen, in der ein Überblick über Preis und Angebot bestimmter Produkte erstellt wird. 
Dadurch wird der tägliche Einkauf für alle Konsumenten erleichtert indem eine Übersicht über alle Ausgaben des Konsumenten entsteht.
\newline
Die Dokumentation fasst die kompletten Phasen der Planung und der Entwicklung mit den jeweiligen Ideen, Tasks und angefertigten Dokumenten zusammen und fungiert als Leitfaden für die alle Projektmitglieder durch das gesamte Projekt.
\newline
Zudem wurde eine Einteilung des Projektes in Definitionsphase, Planungsphase, Durchführungsphase und Abschlussphase als angemessen empfunden und in diesem Dokument angewandt.
Diese Dokumentation ist parallel zur Durchführungsphase entstanden.

\newpage
\section*{Abkürzungsverzeichnis}
\newpage
\section*{Tabellenverzeichnis}

\section{Vorbetrachtung}
Hier in der Vorbetrachtung haben wir verschiedene Unterteilungen vorgenommen. 
Zum einen kann man hier die Aktivitätsliste aller Projektmitglieder anschauen,  wo man sieht welches Teammitglied mit welchen Themen sich befasst hat. 
Desweiteren sieht man hier welche Entscheidungen und Überlegungen hinsichtlich der Abteilungen notwendig waren bis das Projekt durchgeführt werden konnte.

\subsection{Problembeschreibung}
Mit der steigenden Vielfalt an Produkten, dem wachsendem Interesse des Kunden an diesen Produkten und der damit einhergehenden Unübersichtlichkeit über die Produktvielfalt verliert der Kunde schnell den Überblick über die Preise der Produkte und die Ausgaben des Konsumenten. 
Wie in der Einleitung beschrieben, bietet die EinkaufsApp die Möglichkeit einen Überblick über getätigte Einkäufe zu schaffen um vor allem die finanziellen Ausgaben pro Woche, Monat oder Jahr zu tracken und Preise gleicher Produkte von unterschiedlichen Anbietern zu vergleichen.  
Diese App soll zudem noch dabei helfen den finanziellen Überblick zu behalten und eine Hilfe für alle Konsumenten sein, die sich öfter fragen, wo ihre Lieblingsprodukte am günstigsten angeboten werden und wie oft sie diese Produkte im Monat kaufen.
Zusätzlich gibt es die Gruppenfunktion, in der z.B. WG-Mitgliedern, die Möglichkeit, die Ausgaben pro Person zu tracken, was die manuelle Kalkulation am Ende eines Monats erspart. 

\subsection{Entscheidung zur Projektdurchführung}
Unsere EinkaufsApp soll die EANs (European Article Number) der Produkte, die Konsumenten bei ihren Einkäufen in den Warenkorb legen und bezahlen, speichern.
Zudem soll sie es ermöglichen die Kosten auf Personen und Gruppen aufzuteilen und zu managen und eine Auswertung je nach Belieben des Nutzers erstellen.
Unsere App bietet viele Funktionen an, die je nach Benutzer mehr oder weniger angewandt werden können.
Unser Ziel mit der Durchführung des Projektes ist es, eine App zu entwickeln, die die Lösung für das oben genannte Problem darstellt. 
Die Produktvielfalt der verschiedenen Anbieter für Konsumenten wird vereinfacht dargestellt und man sieht auf einen Blick seine Ausgaben.

\newpage

\subsection{Funktionen}
Unsere Ziele des Projektes sind untergliedert in Grundidee und Systemstruktur:
\begin{itemize}
\item[-] Für GPS eine Abfrage an Google Maps implementieren
\item[-] Karte mit Standort verschiedener Läden dem Benutzer anzeigen
\item[-] Automatisch generierte Einkaufsliste
\item[-] GS1 verwaltet alle EANs: 1worldsync soll Schnittstellen für den Zugriff bereitstellen
\item[-] 95 Euro kostet Zugang for 1worldsync um ihre Schnittstelle zu verwenden
\item[-] Facebook, Twitter usw. zur Registrierung verwenden
\item[-] Artikel zuweisen nach Einkauf auf Gruppenmitglieder 
\item[-] was passiert falls kein Empfang vorhanden ist (Offline-Version)
\item[-] Kostenteilung unter Leuten, die nicht in Datenbank und nicht in Gruppe sind und eventuelle Trennung bestimmter Sachen unter den Leuten in der Gruppe z.B. Frauen wollen Sekt, Männer Bier
\item[-] Einkaufsbewertung --> wie gefällt der Laden? Wie zufrieden war man mit dem Einkauf? Alles erhalten? Parkplatz 
vorhanden?
\item[-] Benutzerverwaltung: Profilbild, Account löschen, etc.--> unter Einstellung auf Homescreen
\end{itemize}
\subsubsection*{Ideen: Out of Focus}
\begin{itemize}
\item[-] zwei oder mehr Leute können gleiche Liste verwenden und diese abarbeiten sodass die anderen Änderungen sofort sehen
\item[-] Einkaufsliste vorher schreiben und diese abarbeiten --> gescannte Produkte werden abgehakt
\end{itemize}
\subsubsection{Grundfunktionen der App}
-Informationen über Artikel via EAN aus DB \\
- von aktiven Einkäufen unabhängige, regelmäßige Kosten zu erfassen. \\

\subsubsection{Hauptfunktionen der App}

-Es soll möglich sein für jemand anderen oder eine Gruppe (z.B. WG) etwas einzukaufen. \\
- Marktfindung über GPS \\
-Anzeige versch. Märkte im Umkreis\\
-Login über FB, TWitter o. ä. - >Benutzerverwaltung\\
-Artikelzuweisung nach Einkauf auf Gruppenmitglieder\\

\subsubsection{Zusatzfunktionen der App}
-Einkaufsbewertung\\
-Offline Funktion\\
-automatisch generierte Einkaufsliste\\
-Eine Web Site ist momentan out of scope, wäre aber eine sinnvolle Ergänzung für die Ausgabe von Statistiken und die Benutzerverwaltung.\\
-Auswertung\\
 a) Durchschnittliche Tageskosten eines Zeitraumes (z.B. Woche oder Monat) \\
 b) Maximal oder Minimalpreis innerhalb eines Zeitraumes\\ (z.B. Woche oder Monat) 
 c) Eine Grafik, die den Ausgabenverlauf innerhalb eines Zeitraumes angibt \\
 d) Eine Extrapolation regelmäßig gekaufter Artikel (Ersatz des „Einkaufzettels“) \\
 e) Das persönliche Tracking der allgemeinen Ausgaben \\

\newpage
\subsection{Projektorganisation}
Unsere große Projektgruppe wurde am 02. Oktober 2015 in die Gruppen Dokumentation, Design und Entwicklung aufgeteilt.
Dass Markus Hube unser Gruppenleiter und in diesem Falle zudem Projektmanager werden soll, stand von vornherein fest.
Er schaffte es das Team zu organisieren und den Sinn und Zweck der App zu definieren und zu erklären. Als Projektmanager war er nun auch für die weitere Team- und Projektorganisation zuständig, so sollten alle Mails, die über die EinkaufsApp rund geschrieben wurden auch an ihn adressiert werden.
Als Verantwortlicher für die Qualitätssicherung und Funktionslogik für die Dokumentation wurde Jannis Grohs benannt.
Wichtig für unsere App ist natürlich auch ein geeigneter App-Name und um diesen endgültigen Namen zu finden, gab es ein Brainstorming in dem Tool Slack.

\subsubsection{Kick-Off-Meeting}
Am 02. Oktober  2015 fand unser erstes Meeting mit den genannten Gruppenmitgliedern statt. 
Wir haben dort erstmal eine Gruppenaufteilung vorgenommen und unser Team in die Abteilungen Dokumentation, Design und Entwicklung geteilt. 
In dem ersten Meeting einigten sich die Abteilungen untereinander auf Tools, die sie gerne nutzen möchten und mit denen auch in der weiteren Projektarbeit gearbeitet wird. Alle Tools werden ausführlich im Abschnitt Organisationstools aufgezählt und definiert.
Zudem wurde in diesem ersten Meeting abgestimmt, dass wöchentlich Telefonkonferenzen zum weiteren Vorgehen des Projektes statt finden werden, sodass die genannten Projektziele auch umgesetzt werden können. 
\newpage

\subsubsection{Soll-Analyse}
In dem hier angeführten Kapitel werden konkrete Ziele für das bevorstehende Projekt formuliert, die auf den oben aufgeführten Funktionen der Applikation basieren. (Tabelle der Grundfunktionen, Die Umsetzungsmöglichkeiten, Prioritäten)

\begin{tabular}{|l|c|r|}
\hline
 Funktion & nötige Umsetzungsaspekte & Priorität \\
\hline
Datenbank & Aufstellen einer MongoDB & Prio 1\\
\end{tabular}

\begin{itemize}
 \item[1.1)] 
  Die App soll in Echtzeit die monetären Ausgaben einer Person speichern, sowie ausgewertet wiedergeben.
 \item[1.2)] Hierfür soll es möglich sein:
 \begin{itemize}
 \item[a)] bei einem Einkauf Informationen über einen Artikel von einem Etikett via Barcodescanner einzulesen, beziehungsweise bei bestehender EAN Nummer aus einer Datenbank zu laden und aus diesen Argumenten einen Einkauf zu erstellen
  \item[b)] sonstige Kosten aufzunehmen, die nicht mit einem EAN Code in Verbindung gebracht werden können.
  \item[c)] von aktiven Einkäufen unabhängige, regelmäßige Kosten zu erfassen.
  \end{itemize}
 \item[1.3)] Es soll möglich sein für jemand anderen oder eine Gruppe (z.B. WG) einzukaufen.
 \item[1.4)] Die Daten werden zentral in einer, über das Internet erreichbare, Datenbank gespeichert.
 \item[1.5)] Die App soll primär ein einfaches Front End bereitstellen, um Informationen zu sammeln und zu verwalten
 \item[1.6)] Eine Web Site ist momentan out of scope, wäre aber eine sinnvolle Ergänzung für die Ausgabe von Statistiken und die Benutzerverwaltung.
 \item[1.7)] Die Möglichkeiten der Auswertung sind vielfältig und können in Listen oder Diagrammen dargestellt werden.
 \item[1.8)] Auswertungsbeispiele:
 \begin{itemize}
\item[a)]Ausgaben innerhalb eines bestimmten Zeitraumes (z.B. Woche oder Monat)
 \item[b)] Maximal oder Minimalpreis innerhalb eines Zeitraumes (z.B. Woche oder Monat) 
\item[c)] Eine Grafik, die den Ausgabenverlauf innerhalb eines Zeitraumes darstellt
\item[d)] Eine Extrapolation regelmäßig gekaufter Artikel (Ersatz des „Einkaufzettels“)
\item[e)]Das persönliche Tracking der allgemeinen Ausgaben
\end{itemize}
 \item[1.9)] Außerdem nicht personenbezogene Auswertungen:
 \begin{itemize}
\item[a)]über beliebteste Artikel
\item[b)] beliebteste Märkte
\item[c)] Durchschnittspreise eines Artikels
\end{itemize}
 \item[2.1)] Ein online verfügbarer Server, auf dem seinerseits ein Datenbank Server und ein Web Server läuft
 \begin{itemize}
\item[a)]Als Datenbank-Server wird MongoDB verwendet
\item[b)]Als Web Server wird Apache verwendet
\end{itemize}
\item[2.2)] Auf dem Web Server befindliche PHP-Skripte stellen die Verbindung zur Datenbank her.
\item[2.3)] Aus der Android-App heraus wird mittels HTTP-Post eine Anfrage an die PHP-Skripte geschickte und die Antwort im JSON-Format wieder an die App zurück geschickt.
\end{itemize}
\newpage

\subsubsection{Ist-Analyse}
Was haben wir? Skillliste!
\newpage

\subsubsection{Anforderungsanalyse}
Was benötigen wir? 
\newpage

\subsubsection{Arbeitsplanung}
\subsection{Designer}
\subsubsection*{ToDos der einzelnen Designer:}
\textbf{Florian Graupeter:}
\begin{itemize}
\item[-]Konzeption Grundstruktur
\item[-]Wie ist der Ablauf bei der App Nutzung, welche Sonderfälle müssen an welchen Punkten beachtet werden, wie ist die generelle Struktur
\end{itemize}
\textbf{Moritz Karsten:}
\begin{itemize}
\item[-]Konzeption Funktionalitäten und Informationsfluss einzelner Ansichten
\item[-]welche Knöpfe soll es geben, welche Informationen werden angezeigt, welche Informationen werden zwischen zwei Ansichten ausgetauscht
\end{itemize}
\textbf{Moritz Schaub:}
\begin{itemize}
\item[-]Visuelles Design
\item[-]Farben, Formen und Anordnung, von Butten und Feldern 
\item[-]eventuelles Logodesign + neuer Name für App
\end{itemize}
\textbf{Jannis Grohs:}
\begin{itemize}
\item[-]Quality of Service (Konsistenz in allem beachten und Vorgaben einhalten)
\item[-]Zusammenfassung über Erwartungshorizont und Ausarbeitung zu Fragen
\item[-]Einfachheit und Intuition im Design beachten, zuverlässige Fehlerbehandlung beachten
\end{itemize}
\subsubsection*{ToDos für alle Designer:}
\begin{itemize}
\item[-]Flussdiagramm erstellen und erweitern
\item[-]Erwartungshorizont der App definieren
\item[-]Anfangsstruktur für die Entwickler festlegen
\end{itemize}
\newpage

\subsection{Entwickler}
 Hier fehlen leider noch die einzelnen Aufgabenbereiche von Vielen! --> Unbedingt nachtragen!
\newline
\newline
\textbf{Sebastian Kiepsch:}
\begin{itemize}
\item[-] Einführung in die Technik- Präsentation vorbereiten
\item[-]Modulplan erstellen
\item[-] bereits erstellten Code kommentieren
\end{itemize}
\textbf{Michael Hein:}
\begin{itemize}
\item[-]Erstellung einer Benutzerverwaltung mit User Interface
\item[-] bereits erstellten Code kommentieren
\end{itemize}
\textbf{Eric Sorgalla:}
\begin{itemize}
\item[-] Einführung in die Technik- Präsentation vorbereiten
\item[-] Modulplan erstellen
\item[-] Kanban-Board erstellen
\end{itemize}
\textbf{Viktor Fuchs:}
\begin{itemize}
\item[-] Namenskonventionen ausarbeiten - Variablen und Funktionen
\end{itemize}
\textbf{Florian Schmitt:}
\begin{itemize}
\item[-] HTML-Code der Proto.io- App exportieren und durchgehen
\end{itemize}
\textbf{offene Aufgaben:}
\begin{itemize}
\item[-] aktualisiertes DB Modell erstellen
\item[-] EAN-Kategorien recherchieren und Tabelle füllen
\item[-] neues DB Modell implementieren
\item[-] UML Diagramm erstellen
\end{itemize}

\newpage
\subsection{Dokumentation}
\textbf{Thomas Elias:}
\begin{itemize}
\item[-] Projektplan erstellen 
\item[-] Skillliste der einzelnen Projektmitglieder erstellen
\item[-] Aktivitätsliste aller Projektmitglieder definieren
\item[-] Ausarbeiten allgemeiner Informationen für Handbuch
\end{itemize}

\textbf{Huong Dang:}
\begin{itemize}
\item[-] Projektseitige Dokumentation
\item[-] Dokumentation der Emails, Meetings und Telefonkonferenzen
\item[-] Ausarbeiten der Aktivitäten des Einkaufsprozesses für Handbuch 
\end{itemize}

\textbf{Annika Köstler:}
\begin{itemize}
\item[-] Schreiben der endgültigen Dokumentation
\item[-] Ausarbeiten der Installationsanleitung 
\end{itemize}
\newpage

\subsubsection{Scrum}
Rahmenbedingungen für Profil und Einkaufslisten
\newpage
\subsection{Sicherheit}
\newpage


\section{Systemarchitektur/-landschaft}
\newpage
\subsection{Modell, View, Controller}
\newpage
\subsection{MongoDB}
\newpage
\subsubsection{OpenShift Server}
\newpage
\subsubsection{Basis Tools}
\newpage

\section{Durchführungsphase}
(Idee: Roadmap, s. Flussdiagramm) (Was haben die Designer für die App geplant und was können die Entwickler davon umsetzen) -> nachfolgende Struktur erklären
\subsection*{Einleitung}
\newpage

\subsection{Registrierung}
\subsubsection*{Design}
\subsubsection*{Entwicklung}
\subsubsection*{Dokumentation} --> Probleme beschreiben

\newpage
\subsubsection{Login}
\subsubsection*{Design}
\subsubsection*{Entwicklung}
\subsection{Entwickler}
\textbf{Michael Hein:}
\newline
Aufgabe: Erstellen einer Benutzerverwaltung
\newline
Funktion der Verwaltung:
\begin{itemize}
\item[1.]Registrieren
\begin{itemize}
        \item[a)]Test ob Benutzername/Email bereits vergeben sind
        \item[b)] ob Email wirklich das Format einer Email hat
        \item[c)]Test ob die Passwörter übereinstimmen --> das Passwort wird als Hash in der  Datenbank gespeichert
\end{itemize}
\item[2.]Login/ Logout
\begin{itemize}
\item[a)]Test ob Benutzername/Passwort korrekt
\end{itemize}
\item[3.]Passwort vergessen 
\begin{itemize}
 \item[a)] Sendet Email hinterlegte Mail mit einem Token im Link welches nur eine Stunde gültig ist
\item[b)]Token ist für Passwort zurücksetzen --> Fehler wenn die Email nicht in der DB existiert welches nur eine Stunde gültig ist
\end{itemize}
\item[4.]Passwort zurücksetzen
\begin{itemize}
\item[a)]Überprüfung ob die Passwörter übereinstimmen
\end{itemize}
\item[5.]Funktion zum schützen von Routen
\begin{itemize}
\item[a)]kann in jede Route via „require“ eingebunden und genutzt werden
\item[b)]verhindert, dass man eine Route ohne ein Login sehen kann
 \item[c)]- Diese Anfrage wird auf die Login Seite weitergeleitet
\end{itemize}
\end{itemize}
\subsubsection*{Dokumentation} --> Probleme beschreiben
\newpage

\subsection{Einkauf}
Gruppeneinkauf vereinfacht, Einkauf fortsetzen eingebaut und Passwort rücksetzen
Gruppenverwaltung wurde zu Verwaltung allgemein:  Verwaltung von Gruppen UND KONTAKTE 
Kontaktverwaltung wird noch erstellt"
\subsubsection*{Design}
\subsubsection*{Entwicklung}
\subsubsection*{Dokumentation} --> Probleme beschreiben
\newpage

\subsection{Nutzerverwaltung}
\subsubsection*{Design}
\subsubsection*{Entwicklung}
\subsubsection*{Dokumentation} --> Probleme beschreiben
\newpage

\subsection{Auswertung}
\subsubsection*{Design}
\subsubsection*{Entwicklung}
\subsubsection*{Dokumentation} --> Probleme beschreiben
\newpage

\section{Problemzusammenfassung}
%Hier wird das Ergebnis beschrieben.
\subsection{Usability der App}
\newpage
\subsection{Organisation Projektmanagement}
\newpage
\section{Projektabschluss}
\subsection{Fertiges Produkt}
\subsection{Aussichten}
nicht umgesetzte Ideen --> siehe Excelliste
\newpage
\subsection{Zusammenfassung}
\newpage
\section{Lesson learned}
\newpage
\section*{Quellen}
\addcontentsline{toc}{section}{Quellen}
\subsection*{Internetquellen}
\addcontentsline{toc}{section}{Internetquellen}
\begin{itemize}
\item[1.]Ionic Framework: \url{http://ionicframework.com/}
\item[2.]Ionic Guide: \url{http://ionicframework.com/docs/guide/}
\item[3.]Ionic Getting Started: \url{http://ionicframework.com/getting-started/}
\item[4.]ngCordova - Plugin Seite \url{http://ngcordova.com/}
\item[5.]BarCode Scanner : Plugin \url{hhttp://ngcordova.com/docs/plugins/barcodeScanner/}
\item[6.]Beispiel Projekt: \url{https://github.com/bastisk/suedm}
\item[7.]Editor: \url{http://brackets.io/}
\item[8.]Angular JS-Kurs: \url{https://www.codeschool.com/courses/shaping-up-with-angular-js/}
\item[9.]Tutorial zum Routing: \url{https://scotch.io/tutorials/angular-routing-using-ui-router}
\item[10.]App-Projekt: \url{http://www.mobile2b.de/ablauf-app-projekt/}
\item[11.] Dokumentationshilfe: \url{http://www.tellsbells.de/dokuwebsite/tbdokumentation.pdf}
\item[12.] Dokumentationshilfe: \url{https://www.lecturio.de/magazin/projekte-dokumentieren/}
\item[13.] Open Source mit API über eine einfachen HTTP-GET-Reguest: \url{http://www.opengtindb.org/api.php}
\item[14.] Suchmaschine der Firma die GTIN-Nummern verwaltet: \url{http://www.gepir.de/v31/V31_client/gtin.aspx}
\end{itemize}
\newpage
\section*{Anhang}
\addcontentsline{toc}{section}{Anhang}
\newpage
\section*{Glossar}
\addcontentsline{toc}{section}{Glossar}
\newpage
\section*{Abbildungsverzeichnis}
\newpage
\section*{Organisationstool- Übersicht}
\begin{itemize}
\item[-]Allgemeine Ablage: GitHub
\item[-]Diskussionsrunden: Slack
\item[-]Informationsaustausch: via Email
\item[-]Diagramme zeichen: via Dia 
\item[-]Kreieren von Web-Prototypen: proto.io
\item[-]Datenbanken und Datenbankenverwaltung: MongoDB, RoboMongo
\end{itemize}
\end{document}

    Status API Training Shop Blog About Pricing 

    © 2015 GitHub, Inc. Terms Privacy Security Contact Help 
